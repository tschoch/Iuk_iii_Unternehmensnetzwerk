\documentclass[12pt,a4paper,titlepage]{article}
\usepackage{graphicx}
\usepackage[ngerman]{babel} 
\usepackage[ansinew]{inputenc} 
\usepackage{setspace}
\usepackage{tabularx}
\usepackage[a4paper,left=2cm,right=2cm,top=3cm,bottom=3cm]{geometry}
\usepackage{fancyhdr}
\pagestyle{fancy} 
\fancyhf{} 
\renewcommand{\headrulewidth}{0.4pt} 
\renewcommand{\footrulewidth}{0.4pt}
\renewcommand{\sectionmark}[1]{\markright{#1}}
\renewcommand{\subsectionmark}[1]{}
\renewcommand{\subsubsectionmark}[1]{}
\lhead{\nouppercase{\rightmark}}
\rhead{\thepage}
\lfoot{Benjamin Mosberger, Tobias Schoch}

%--------Titelseite----------
\begin{document}
\newgeometry{left=0cm,right=0cm, top=1cm} 
\thispagestyle{empty}
\begin{center}
	\includegraphics[height=2cm]{logo_ntb.png}
	\hspace*{6cm}
	\includegraphics[height=1.6cm]{logo_htw.png}\\
	\vspace{5cm}
	\Large{NTB - Interstaatliche Hochschule f�r Technik Buchs}\\
	\vspace{3cm}
	\Huge{IuK\_III\_U-Konzeption und Aufbau eines Unternehmensnetzwerkes}\\
	\vspace{8cm}
	\Large{}
	\doublespacing
	\begin{tabular}{lll}
		\textbf{Studierende:} & Benjamin Mosberger, &Tobias Schoch \\ 
		\textbf{Dozent:} & Beat Bigger \\
		\textbf{Datum:} & Herbstsemester 2015
	\end{tabular}	
\end{center}
\restoregeometry
\pagebreak

%--------Abstract----------
\setcounter{page}{1}
\onehalfspacing 

\section{Zusammenfassung}
Die Aufgabe dieser Projektarbeit ist ein Unternehmensnetzwerk im Labor der HTW Chur  zu planen, aufzubauen und zu Dokumentieren. 

\vspace{-1,2em}

\section*{Abstract}
Das ganze auf Englisch.
\pagebreak

%--------Inhaltsverzeichnis----------
\tableofcontents 
\pagebreak

%--------Inhalt----------
\section{Ausgangslage} 
\subsection{Fallbeispiel } 
\subsection{Praktikumsausr�stung} 
\section{Konzept}
\subsection{IP Addresskonzept } 
\subsection{Routingkonzept} 
\subsection{Securitykonzept} 
\subsubsection{ACL}
\subsubsection{Layer 2 Security}
\subsection{Serverservices}
\subsection{Netzwerkplan}
\section{Planung}
\section{Umsetzung}
\end{document}







